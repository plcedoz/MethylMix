%
% NOTE -- ONLY EDIT THE .Rnw FILE!!!  The .tex file is
% likely to be overwritten.
%
% \VignetteIndexEntry{MethylMix}
% \VignetteDepends{}
% \VignetteKeywords{MethylMix, Clustering, DNA methylation}
% \VignettePackage{MethylMix}

\documentclass[11pt]{article}
\RequirePackage{/Library/Frameworks/R.framework/Versions/3.1/Resources/library/BiocStyle/sty/Bioconductor}

\AtBeginDocument{\bibliographystyle{/Library/Frameworks/R.framework/Versions/3.1/Resources/library/BiocStyle/sty/unsrturl}}\usepackage{amsmath,epsfig,fullpage}
\usepackage[authoryear,round]{natbib}
\usepackage{hyperref}

\parindent 0in
\bibliographystyle{abbrvnat}

\usepackage{Sweave}
\begin{document}
\Sconcordance{concordance:MethylMix.tex:MethylMix.Rnw:%
1 9 1 1 2 1 0 1 2 7 1 1 0 21 1 1 2 1 0 1 1 3 0 1 2 65 1 1 2 1 0 2 1 3 0 1 2 5 1 %
1 2 1 0 2 1 3 0 1 2 8 1 1 2 1 0 2 1 3 0 1 2 9 1 1 2 1 0 4 1 3 0 1 2 11 1 1 2 1 0 %
3 1 3 0 1 2 3 1 1 2 4 0 1 2 12 1 1 2 1 0 2 1 3 0 1 2 8 1 1 2 1 0 1 2 1 0 1 1 3 0 %
1 2 6 1 1 2 1 0 4 1 1 18 17 0 2 1 3 0 1 2 5 1 1 3 2 0 1 2 4 0 1 2 5 1 1 2 14 0 1 %
2 4 1}


\title{MethylMix\\ An R package for identifying DNA methylation driven genes}
\author{Olivier Gevaert}
\maketitle
\begin{center}
Stanford Center for Biomedical Informatics\\
Department of Medicine\\
1265 Welch Road\\
Stanford CA, 94305-5479\\
\end{center}
%\tableofcontents

%%%%%%%%%%%%%%%%%%%%%%%%%%%%%%%%%%%%%%%%%%%%%%%%%%%%%%%%%%%%%%%%%%%%%%%%%%%

%%%%%%%%%%%%%%%%%%%%%%%%%%%%%%%%%%%%%%%%%%%%%%%%%%%%%%%%%%%%%%%%%%%%%%%%%%%
\section{Getting started}

{\bf Installing the package.} To install the \Rpackage{MethylMix} package, 
the easiest way is through bioconductor: \\

\begin{Schunk}
\begin{Sinput}
> source("http://bioconductor.org/biocLite.R")
> biocLite(MethylMix)
\end{Sinput}
\end{Schunk}
Other ways to install MethylMix is to first download the appropriate file for 
your platform from the Bioconductor website \url{http://www.bioconductor.org/}. 
For Windows, start R and select the \texttt{Packages} menu, then \texttt{Install 
package from local zip file}.  Find and highlight the location of the zip file 
and click on {\tt open}. For Linux/Unix, use the usual command \texttt{R 
CMD INSTALL} or install from bioconductor.\\

{\bf Loading the package.} To load the \Rpackage{MethylMix} package in your R 
session, type \texttt{library(MethylMix)}.\\

{\bf Help files.}  Detailed information on \Rpackage{MethylMix} package 
functions can be obtained in the help files. For example, to view the help file 
for the function \texttt{MethylMix} in a Rsession, use \texttt{?MethylMix}.\\


%%%%%%%%%%%%%%%%%%%%%%%%%%%%%%%%%%%%%%%%%%%%%%%%%%%%%%%%%%%%%%%%%%%%%%%%%%%
\section{Introduction}
DNA methylation is a mechanism whereby a methyl-group is added onto a CpG site. 
Methylation of these CpG sites is associated with gene silencing and is an 
important mechanism for normal tissue development and is often involved in 
diseases such as cancer. Recently, many high throughput data has been generated 
profiling CpG site methylation on a genome wide bases. This has created large 
amounts of data on DNA methylation for many disease. Computational analysis of 
DNA methylation data is required to identify potentiall aberrant DNA methylation 
compared to normal tissue. MethylMix was developed to tackle this question using 
a computational approach. MethylMix identifies differential and functional DNA 
methylation by using a beta mixture model to identify subpopulations of samples 
with different DNA methylation compared to normal tissue. Functional DNA 
methylation refers to a significant negative correlation based on matched gene 
expression data. Together MethylMix outputs hyper and hypomethylated genes which 
can be used for downstream analysis, and are called MethylMix drivers. MethylMix 
was designed for cis-regulated promoter differential methylation and works best 
when specific CpG sites are profiled associated with a gene. For example using 
data from the 27k and 450k Infinium platforms. 

\section{Data access and preprocessing}
The data in this vignette is accessible at The Cancer Genome Atlas (TCGA) 
portal. A programmatic way of donwloading data is through the firehose\textunderscore get 
tool developed by the broad institute (\hyperref[label_name]
{''https://confluence.broadinstitute.org/display/GDAC/Download''}). Firehose\textunderscore get
provides a unified way to download data for all cancer sites and all platforms.
We suggest to only use Level3 data for users unfamiliar with the TCGA data. For 
MethylMix two data types are relevant, DNA methylation data and gene expression 
data. The methylation data is provided using two platforms the 27k and 450k 
Illumina platform. We suggest to start from the methylation files marked as 
\emph{Merge\textunderscore methylation\textunderscore \textunderscore 
humanmethylation}. The methylation archives for both platforms contain the
beta-values for each sample and gene. For gene expression there are also two options, either 
microarray data \emph{Merge\textunderscore transcriptome\textunderscore 
\textunderscore agilentg4502a} or RNA sequencing data \emph{Merge\textunderscore 
rnaseqv2\textunderscore \textunderscore illuminahiseq\textunderscore rnaseqv2}.\\
The only preprocessing we recommend for both data sets is to correct for batch 
effects. We use Combat to adjust for batch effects. One can either donwload the 
Combat function from the authors website, or use the \emph{sva} bioconductor 
package which contains the Combat script. The batch information for each sample 
can be found in the Biospecimen Metadata Browser at the TCGA data portal 
\hyperref[TCGAbatch]{''https://tcga-data.nci.nih.gov/uuid/uuidBrowser.htm''}.
Next to take care of the multiple probes that are available for each gene for 
the methylation data, we use a clustering approach that is explained in section 
\ref{CpGanotation}.

\section{Data input for MethylMix}
To run MethylMix at least a methylation data set of a particular disease is 
required. This will allow to identify methylation states associated with a 
disease for each gene of interest. 

\begin{Schunk}
\begin{Sinput}
> library(MethylMix)
> data(METcancer)
> head(METcancer)
\end{Sinput}
\end{Schunk}

Adding an appropriate normal or baseline methylation data set increases 
MethylMix functionality significantly by also being able to distinguish 
between hyper or increased methylation vs. hypo or decreased methylation with 
respect to the normal or baseline methylation data set. 

\begin{Schunk}
\begin{Sinput}
> library(MethylMix)
> data(METnormal)
> head(METnormal)
\end{Sinput}
\end{Schunk}

Finally, if matched gene expression data is available for the same samples that 
methylation data was available studying this disease, MethylMix will also 
identify functional differnential methylation by focusing only on differentialy 
methylation that has a significant inversely correlated effect with gene 
expression. Each of these three data sets are matrix objects with genes in the 
rows with unique rownames (e.g. gene symbols) and samples or patients in the 
columns with unique patient names. 

\begin{Schunk}
\begin{Sinput}
> library(MethylMix)
> data(MAcancer)
> head(MAcancer)
\end{Sinput}
\end{Schunk}

\section{Annotation of CpG probes}\label{CpGanotation}
When only probe level Illumina data is available, mapping probes to genes is 
recommended before building mixture models. This allows to focus on cis-
regulated differential methylation by only focusing on differential methylation 
of CpG sites to their closest gene transcripts. Both the 27k and 450k Illumina 
platforms have database R packages that provide the necessary mapping 
information. The next example maps the probes of the 27k Illumina platform to 
Gene Symbol ids. 

\begin{Schunk}
\begin{Sinput}
> library(IlluminaHumanMethylation27k.db)
> ProbeToSymbol <- IlluminaHumanMethylation27kSYMBOL
> mapped_probes <- mappedkeys(ProbeToSymbol)
> mapped_probes_List <- as.list(ProbeToSymbol[mapped_probes])
> mapped_probes_List[3:4]
\end{Sinput}
\end{Schunk}

Other packages can be used as well such as the FDb.InfiniumMethylation.hg19 
package, which provides annotation for both 27k and 450k but requires a 1GB 
download. \\

%%%%%%%%%%%%%%%%%%%%%%%%%%%%%%%%%%%%%%%%%%%%%%%%%%%%%%%%%%%%%%%%%%%%%%%%%%%
\section{MethylMix on glioblastoma TCGA data from the 27k Infinium platform}

An example data set from the glioblastoma TCGA project is available in 
MethylMix. The DNA methylation data for cancer and normal samples from the 27k 
Infinium platform, and matched gene expression data can be loaded as follows: 

\begin{Schunk}
\begin{Sinput}
> library(MethylMix)
> data(METcancer)
> data(METnormal)
> data(MAcancer)
\end{Sinput}
\end{Schunk}

Next, we can run MethylMix to identify hypo and hypermethylated genes as 
follows:

\begin{Schunk}
\begin{Sinput}
> MethylMixResults = MethylMix(METcancer,METnormal,MAcancer)
\end{Sinput}
\end{Schunk}



%%%%%%%%%%%%%%%%%%%%%%%%%%%%%%%%%%%%%%%%%%%%%%%%%%%%%%%%%%%%%%%%%%%%%%%%%%%
\section{MethylMix on breast cancer data for CDH1 TCGA data from the 450k 
Infinium platform}

MethylMix can also be applied on Infinium 450k data. To illustrate this we 
extracted DNA cancer and normal methylation data for a single gene for the 
breast cancer TCGA project. We focused on CDH1 differential methylation. 
First the cancer and normal DNA methylation for CDH1 is loaded and the matched 
gene expression data: 

\begin{Schunk}
\begin{Sinput}
> data(METcancer_CDH1)
> data(METnormal_CDH1)
> data(MAcancer_CDH1)
\end{Sinput}
\end{Schunk}

On the 450k platform many more probes are available for every gene. MethylMix 
can be run on each probe independently however, this will require significant 
computational resources and many probes show correlated methylation profiles. 
Therefore, a clustering algorithm can be used to reduce the number of probes and 
create CpG clusters with similar methylation profiles. For example, the gene 
CDH1 has 63 probes in the breast cancer TCGA data and using hierarchical 
clustering we can cluster these probes into CpG clusters as follows:

\begin{Schunk}
\begin{Sinput}
> ProbeCorrelation=cor(t(METcancer_CDH1),method='pearson')
> ClusterResults=hclust(as.dist(1-ProbeCorrelation), method = "complete", 
+                       members = NULL)
> plot(ClusterResults)
\end{Sinput}
\end{Schunk}

We used hierarchical culstering with complete linkage and the 1- pearson 
correlation coefficient as a distance measure. Next, we can use a cutoff of 
corresponding to a minimum correlation of 0.3 in each cluster to define CpG 
clusters. Then we summarize the clusters by taking the average of the tightly 
correlated probes in each cluster:

\begin{Schunk}
\begin{Sinput}
> METcancer_CDH1_Clustered=matrix(0,0,length(colnames(METcancer_CDH1)))
> METnormal_CDH1_Clustered=matrix(0,0,length(colnames(METnormal_CDH1)))
> Clusternames=c()
> Clusters=cutree(ClusterResults,h=0.7)
> GeneProbes=rownames(METcancer_CDH1)
> for (i in 1:length(unique(Clusters))) {
+      tmpGeneProbes=GeneProbes[Clusters==i]
+      if (length(tmpGeneProbes)>1) {
+           tmpAveragedProfile=colMeans(METcancer_CDH1[tmpGeneProbes,])
+           METcancer_CDH1_Clustered=rbind(METcancer_CDH1_Clustered,
+                                          tmpAveragedProfile)          
+           # Same for normal
+           tmpAveragedProfile=colMeans(METnormal_CDH1[tmpGeneProbes,])
+           METnormal_CDH1_Clustered=rbind(METnormal_CDH1_Clustered,
+                                          tmpAveragedProfile)
+      } else {          
+           METcancer_CDH1_Clustered=rbind(METcancer_CDH1_Clustered,
+                                          METcancer_CDH1[tmpGeneProbes,])
+           METnormal_CDH1_Clustered=rbind(METnormal_CDH1_Clustered,
+                                          METnormal_CDH1[tmpGeneProbes,])
+      }
+      Clusternames=c(Clusternames,paste('CDH1---Cluster',i,sep=""))
+ }
> rownames(METcancer_CDH1_Clustered)=Clusternames
> rownames(METnormal_CDH1_Clustered)=Clusternames
\end{Sinput}
\end{Schunk}

For CDH1 in the TCGA breast cancer data, this results in seven CpG clusters, 
reducing the dimensionality nine fold from the original 63 probes. Now we can 
build a MethylMix model for each of these seven clusters and investigate which 
CDH1 CpG cluster is functionally and differentially methylated in breast cancer. 

\begin{Schunk}
\begin{Sinput}
> MethylMixResults=MethylMix(METcancer_CDH1_Clustered,METnormal_CDH1_Clustered,
+                            MAtumor_CDH1,OutputRoot='',Parallel=FALSE)
> MethylMix_PlotModel('CDH1---Cluster3',METcancer_CDH1_Clustered,MethylMixResults,
+                     MAtumor_CDH1,METnormal_CDH1_Clustered)
\end{Sinput}
\end{Schunk}

It turns out that one CDH1 cluster, cluster 3, is hypomethylated in 40\% of 
breast cancer samples. 

\section{Session Information}

\begin{Schunk}
\begin{Sinput}
> toLatex(sessionInfo())
\end{Sinput}
\begin{itemize}\raggedright
  \item R version 3.1.1 (2014-07-10), \verb|x86_64-apple-darwin13.1.0|
  \item Locale: \verb|C/en_US.UTF-8/en_US.UTF-8/C/en_US.UTF-8/en_US.UTF-8|
  \item Base packages: base, datasets, grDevices, graphics, methods, stats,
    utils
  \item Other packages: MethylMix~0.99.1
  \item Loaded via a namespace (and not attached): BiocStyle~1.2.0,
    RColorBrewer~1.0-5, RPMM~1.10, cluster~1.15.2, codetools~0.2-9,
    doParallel~1.0.8, foreach~1.4.2, iterators~1.0.7, optimx~2013.8.6,
    parallel~3.1.1, tools~3.1.1
\end{itemize}\end{Schunk}


%%%%%%%%%%%%%%%%%%%%%%%%%%%%%%%%%%%%%%%%%%%%%%%%%%%%%%%%%%%%%%%%%%%%%%%%%%%
\end{document}
 
